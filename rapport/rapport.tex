\documentclass[a4paper]{article}
\usepackage[utf8]{inputenc}
\usepackage[french]{babel}
\usepackage[T1]{fontenc}
\usepackage{amsmath}
\usepackage{amssymb}
\usepackage{amsthm}
\usepackage{listings}
\usepackage{enumitem}
\usepackage{relsize}
\usepackage{dsfont}
\usepackage{graphicx}
\usepackage[margin=1in]{geometry}

\newcommand{\K}{\ensuremath\mathbb{K}}
\newcommand{\N}{\ensuremath\mathbb{N}}
\newcommand{\Z}{\ensuremath\mathbb{Z}}
\newcommand{\Q}{\ensureFmath\mathbb{Q}}
\newcommand{\R}{\ensuremath\mathbb{R}}
\newcommand{\U}{\ensuremath\mathbb{U}}
\newcommand{\C}{\ensuremath\mathbb{C}}
\newcommand{\E}{\ensuremath\mathbb{E}}
\newcommand{\V}{\ensuremath\mathbb{V}}
\renewcommand{\P}{\ensuremath\mathbb{P}}

\renewcommand{\(}{\left(}
\renewcommand{\)}{\right)}

\newcommand{\la}{\leftarrow}
\newcommand{\xla}{\xleftarrow}
\newcommand{\ra}{\rightarrow}
\newcommand{\xra}{\xrightarrow}

\renewcommand\labelitemi{---}

\setlength\parindent{0pt}

\newtheorem*{definition}{Définition}
\newtheorem*{theorem}{Théorème}
\newtheorem*{algo}{Algorithme}
\renewcommand*{\proofname}{Preuve}

\title{Rapport: ReactiveRS2}
\author{Thibaut Pérami, Mathieu Fehr}

\begin{document}

\maketitle

\section{Introduction}

Pour ce projet deSignalRuntimeRef programmation parallèle et réactive, nous avons d'abord écrit
une première version (qui n'a pas été finie), en suivant la structure proposée
dans les TP.

Cependant, nous avons ensuite décidé de changer la structure de la bibliothèque.
Nous pensons que cette nouvelle structure permet d'être plus efficace, et offre
une représentation des programmes réactifs plus intuitive. Cependant, cette
nouvelle structure est plus difficile à programmer (en Rust), et pose quelques
problèmes pour l'écriture des fonctions and\_then et flatten (que nous n'avons
pas codés par soucis de temps).

\section{Fonctionnement de la bibliothèque}

La bibliothèque se repose non pas sur un vecteur de continuations, mais sur un
graphe. 

\subsection{Noeuds}

Un noeud est essentiellement une opération élémentaire
réactive. L'éxecution d'un noeud peut être l'éxecution d'une fonction non
réactive (comme un FnMut ou un FnOnce), une pause, une emission d'un signal, un
await d'un signal, etc... Un noeud possède une entrée et une sortie, afin de
passer des valeurs dans le programme.

Pour stocker des noeuds dans le graphe, il faut qu'ils aient le même type. Pour
cela, nous ne stockons que des noeuds prenant en entrée () et en sortie ().
Cela nous oblige à passer par des Rc<Cell> ou des Arc<Mutex> pour passer des
valeurs, ce qui ralentit l'éxecution du code. Pour remédier à ce problème, nous
pouvons associer deux noeuds séquentiellement dans un même noeud, pour pouvoir
par exemple appliquer la fonction f(g(x)) dans un même noeud, au lieu de passer
g(x) dans un Rc/Arc, pour le passer au noeud contenant f.

Pour savoir quels noeuds sont éxecuter, nous stockons simplement dans un tableau
la liste des noeuds à éxecuter à l'instant courant, et à l'instant suivant.

Pour simplifier l'écriture de ce graphe, l'utilisateur n'a pas à utiliser les
noeuds directement, et utilise une autre structure: les processus.

\subsection{Processus}
Un processus permet à l'utilisateur de compiler une petite opération
réactive en plusieurs noeuds. Cela permet de ne pas toujours réécrire les mêmes
blocs, et d'avoir une compilation efficace vers le graphe de noeuds.

Un processus possède aussi une entrée, et une sortie. Il y à deux types de
processus: les processus immédiats et les processus non immediats. Les processus
immédiats se compilent en un unique noeud, tandis que les processus non
immédiats se compilent en plusieurs noeuds, dont un d'entrée et de sortie.
L'utilité de cette structure est que plusieurs processus immédiats compilés
séquentiellement peuvent se compiler en un unique noeud, contenant la séquence
de tout les noeuds immédiats, et réduisant ainsi le nombre de Rc/Arc utilisés.
Aussi, un processus non immédiat en les noeuds (Nentrée, Nmilieu,
Nsortie), mis en séquence avec un noeud non immédiat (Nimmédiat), a son noeud
Nsortie compilé dans un même noeud avec Nimmédiat. Cela permet d'avoir un
minimum de Rc/Arc utilisés.

\subsection{Signaux}

Les signaux utilisent l'attente passive pour pouvoir effectuer les actions. Cela
permet de ne pas avoir à effectuer des calculs pour un signal lorsque celui-ci
n'est pas utilisé. Pour cela, on garde en mémoire les deux derniers instants
dans lesquels un signal à été émis, et on ne met a jour le signal que lorsqu'on
intéragit avec lui (a travers un emit, ou lorsqu'on récupère sa valeur).

\subsection{Macros}

Pour simplifier l'écriture d'un programme réactif, nous avons utilisé des macros
procédurales. Ces macros, écritent dans un crate à part, permettent d'avoir une
syntaxe plus simple, en réduisant l'imbrication. La macro parcours l'arbre de
syntaxe reçue par le compilateur, et effectue des modifications à cet arbre pour
obtenir le code Rust souhaité.

\section{Améliorations possibles}

Par manque de temps, du à des complications avec Rust, nous n'avons pas pu
écrire certaines structure. Nous présentons ici des idées d'implémentations pour
ces structures:

\begin{itemize}
\item Flatten: Puisque la bibliothèque se base sur une compilation du programme
  réactif, il est difficile d'implémenter des processus produits dynamiquement.
  Cependant, nous avons une idée d'implémentation. L'idée serait d'avoir un
  noeud prenant en entrée un processus, et compilant se processus, pour contenir
  un sous-graphe. Le noeud contiendrait aussi un runtime, pour savoir quels
  noeuds sont à éxecuter.
\end{itemize}

\section{Exemple de compilation de noeuds}

% TODO Afficher des graphes ici.

\section{Exemple d'application}

En idée d'application simple hautement parallèle, nous avons pensé au jeu de la
vie. Pour utiliser au maximum l'aspect réactif, nous avons utilisé des signaux
pour faire communiquer les cases entre elles. Chaque case contient un signal, et
lorsque la case est allumée, elle envoie un signal a toute ses cases voisines.
La réception des signaux permet alors de calculer l'étape suivante. La
structure, étant sparse, n'a pas besoin de calculer la valeur d'une case lorsque
celle ci ne possède pas de voisins allumés. Pour afficher un état, nous
utilisons un signal qui est émis pour chaque case allumée. Ce signal contient
donc l'état du tableau.

\section{Liste (non exhaustive) de problèmes avec Rust}
\subsection{Bug Luc}
-Bug (qu'a trouvé Luc) qui a été corrigé depuis.

\subsection{Trait A ou B}
-On ne peut pas dire qu'un trait est soit A, soit B.

\subsection{Inférence de type}
-La compilation prends 5 minutes car l'inférence de types est broken.

\end{document}